\documentclass[a4paper,12pt]{article}
\usepackage{a4wide}
\usepackage{enumitem}
\usepackage{titling}
\usepackage{textcomp}
\usepackage{amsmath}
\usepackage{dsfont}
\usepackage{amssymb}
\usepackage{adjustbox}
\usepackage{hyperref}
\usepackage{indentfirst}

\setlength{\droptitle}{-8em}
\newcommand\tab[1][1cm]{\hspace*{#1}}



\begin{document}
	\title{Kin Recognition Using Weighted Graph Embeddings - Progress Report}
	\author{Manu Varma : \href{mailto:mv465@cam.ac.uk}{mv465@cam.ac.uk} \\ Supervisor : Daniel Bates \\ Director of Studies : Robert Mullins \\ Overseers : Frank Stajano and Amanda Prorok}
	\date{\vspace{-10ex}}
	\maketitle	\ 

% An indication of what work has been completed and how this relates to the timetable and work plan in the  original proposal. The progress report should answer the following questions:
% Is the project on schedule and if not, how many weeks behind or ahead?
% What unexpected difficulties have arisen?
% If the project is behind, what actions have been taken to address this and when will progress be back on track?
% Briefly, what has been accomplished?
% It should be possible to understand the progress report independently of the original proposal, thus ‘I have completed implementing the wombat module’ rather than ‘I have completed points 1 and 3 in the proposal but not point 2’.
% In straightforward cases (entirely on schedule), one side of A4 could suffice. If the project is in difficulties, a new work plan should be included.

\section{Work Accomplished}

I was able to implement the main algorithm, WGEML, from the paper, ``Weighted Graph Embedding-Based Metric
Learning for Kinship Verification". This included creating the functions that would get the necessary face descriptors, Local Binary Patterns, Histogram of Gradients, Scale-Invariant Feature Transform and a VGG face descriptor, from each image. 

Furthermore, I implemented unit tests along the way to make sure that my functions were running as I expected them to run. I managed to get a code coverage of $97\%$ for my unit tests, which excludes any scripts that were written. 

\section{Schedule}

The project is currently on schedule. At this point in my schedule, I am supposed to be working on ablation studies which is what is currently happening. However, the original schedule had a few problems where I overestimated the amount of time to implement the penalty and intrinsic graphs but I also didn't schedule any time to write the scripts needed to run it end-to-end or any of the necessary data preparation functions to get the data into a format that is consistent with how I wrote my functions. Furthermore, I also didn't schedule any time to obtain any of the results. However, in the end, the amount of work I didn't schedule was done in the amount of time I overestimated in my schedule which means that I am still on track, even though the schedule partially changed in the contents. 

\section{Unexpected Difficulties}

The main unexpected difficulties came from taking the data in the form that was downloaded and transforming it into the way I wanted it to be. % TODO: Explain which parts

Other unexpected difficulties occurred from the paper I was implementing being vague on certain parts of the project. % TODO: Explain which parts

\end{document}