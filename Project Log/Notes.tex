\documentclass[a4paper,12pt]{article}
\usepackage{a4wide}
\usepackage{enumitem}
\usepackage{titling}
\usepackage{alltt}
\usepackage{amsmath}
\usepackage{amssymb}
\setlength{\droptitle}{-8em}
\newcommand\tab[1][1cm]{\hspace*{#1}}

\begin{document}
	\title{Project Log on WGEBML Kin Recognition}
	\date{\vspace{-5ex}}
	\maketitle	\ 
	
\section{Related Work}

\subsection{Main Paper - WGEML}

\begin{itemize}
	\item The main parts of the paper are the face detection, the four face descriptors: LBP, HOG, SIFT, VGG, the penalty graphs and intrinsic graph and then using the graphs to figure out how the faces in the images are related. 
\end{itemize}

\section{Implementation Notes}


\subsection{Face Detection}
\begin{itemize}
	\item Firstly, OpenCV2 was used to create a base implementation to draw a rectangle around a person's face in an image. This was done using the pre-trained classifier in "haarcascade\_frontalface\_default.xml". This allowed us to take a file image and output another saved file image which was the original picture with a rectangle around each face. 
\end{itemize}


\end{document}