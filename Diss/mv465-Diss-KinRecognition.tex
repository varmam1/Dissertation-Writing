
\documentclass[11pt,a4paper,twoside]{report}
\usepackage[pdfborder={0 0 0}]{hyperref}    % turns references into hyperlinks
\usepackage[margin=20mm]{geometry}  % adjusts page layout
\usepackage{graphicx}  % allows inclusion of PDF, PNG and JPG images
\graphicspath{ {./images/} }
\usepackage{verbatim}
\usepackage{amsmath}
\usepackage{amssymb}
\usepackage{dirtree}
\usepackage{mathtools}
\DeclarePairedDelimiter\paren{\lparen}{\rparen}
\DeclarePairedDelimiter\bracks{\lbrack}{\rbrack}
\usepackage{xcolor}
% \usepackage{docmute}   % only needed to allow inclusion of proposal.tex

\raggedbottom                           % try to avoid widows and orphans
\sloppy
\clubpenalty1000%
\widowpenalty1000%
\setcounter{secnumdepth}{4}
\setcounter{tocdepth}{4}

\renewcommand{\baselinestretch}{1.1}    % adjust line spacing to make
                                        % more readable

\begin{document}

%%%%%%%%%%%%%%%%%%%%%%%%%%%%%%%%%%%%%%%%%%%%%%%%%%%%%%%%%%%%%%%%%%%%%%%%
% Title

\pagestyle{empty}

\rightline{\LARGE \textbf{Manu Varma}}

\vspace*{60mm}
\begin{center}
\Huge
\textbf{Kin Recognition Using Weighted Graph Embeddings} \\[5mm]
Computer Science Tripos -- Part II \\[5mm]
St John's College \\[5mm]
\today  % today's date
\end{center}

%%%%%%%%%%%%%%%%%%%%%%%%%%%%%%%%%%%%%%%%%%%%%%%%%%%%%%%%%%%%%%%%%%%%%%%%%%%%%%
% Proforma, table of contents and list of figures
\let\cleardoublepage\clearpage
\pagestyle{plain}

\newpage
\section*{Declaration}

I, Manu Varma of St. John's College, being a candidate for Part II of the Computer
Science Tripos, hereby declare
that this dissertation and the work described in it are my own work,
unaided except as may be specified below, and that the dissertation
does not contain material that has already been used to any substantial
extent for a comparable purpose.

\bigskip
\leftline{Signed Manu Varma}

\medskip
\leftline{Date \today}

\chapter*{Proforma}

{\large
\begin{tabular}{ll}
Name:               & \bf Manu Varma                       \\
College:            & \bf St John's College                     \\
Project Title:      & \bf Kin Recognition Using Weighted Graph Embeddings \\
Examination:        & \bf Computer Science Tripos -- Part II, June 2021  \\
Word Count:         & \bf TBA  \\ % TODO: Change
Project Originator: & \bf The Dissertation Author                   \\
Supervisor:         & \bf Daniel Bates                    \\ 
\end{tabular}
}
\section*{Original Aims of the Project}

\section*{Work Completed}

\section*{Special Difficulties}

\tableofcontents

\listoffigures


\pagestyle{headings}

%%%%%%%%%%%%%%%%%%%%%%%%%%%%%%%%%%%%
%%%%%%%%%%% Introduction %%%%%%%%%%%
%%%%%%%%%%%%%%%%%%%%%%%%%%%%%%%%%%%%

% Intro comments:
% * It might be useful to have a diagram to illustrate the problem. e.g. [image, image, "father-daughter"] -> ✓
% * You will probably need to define terms like CNN and SVM the first time you use them

\chapter{Introduction}
% TODO: A stronger hook sentence is definitely needed here. 

Kinship recognition is the way of recognizing whether two given people are related to each other or not based on how they look. This is an evolutionary trait in humans as it is advantageous to inclusive fitness for an organism to be able to recognize which of their neighbors were close relatives \cite{HamiltonGenetic}. Thus, it stands to reason that the ability to recognize kinship relationships has evolved in humans. 
In humans, specifically, facial resemblance is expected to serve as an indicator of kinship as we have seen that strangers are able to match photographs of mothers to their infants without any prior contact with any of the family \cite{mateo2015}. 

Computational kinship recognition is the field of computationally figuring out kinship relationships between people without any prior knowledge of the family. 

\section{Problem Overview}

There are multiple problems in the field of computational kin recognition, in which case I will focus on one of them. The first of which takes as input a pair of images and a proposed kinship relationship, for example father-daughter, and recognizes whether the relationship exists. 
These relationships tend to be more commonly parent-child relationships as opposed to sibling-sibling relationships, for example, which are less commonly used. A further extension of the main kin recognition problem is Tri-Subject Kinship which takes 3 images, two parents and a child, and determines if they are related 
or not. These problems are the ones that are being looked at in this dissertation. 

However, some other major problems in the field include search-and-retrieval and family classification. Search-and-retrieval is the problem which takes an image of a person as input and searches through a database to find people with whom they are most likely to be related to. This outputs a list of these people that they could be related to. 
Family classification deals with a similar task of taking an image of a person as input and figuring out which family they may belong to. 

\section{Motivation}

Accurate kinship recognition software has multiple applications in humanitarian issues. 
With regards to humanitarian issues, by using kin recognition, we can better recognize missing children and match them with their parents \cite{robinson2020visual}.
It can also be used for stopping human traffickers from claiming they are family members of the victim or to reunite families across refugee camps. 

Furthermore, there are potential use-cases in social media and also poses privacy protections. 
% TODO: Strengthen up the motivations here 
\section{Related Work}     

One of the first papers in the field of computational kinship recognition used color, facial parts, facial distances and a Histogram of Gradients vector as the features for each image. 
Using these features, K-Nearest-Neighbors and an SVM were used in order to classify the image pairs into true and false parent-child pairs \cite{fang2010}. A classification accuracy of $70.67\%$ was obtained overall and an SVM with a radial basis kernel obtained an accuracy of $68.6\%$. 

Other approaches include using deep learning to solve the problem. One paper used a Siamese CNN approach by putting both of the face images in the pair through a SqueezeNet network which was trained on VGGFace2 which creates a feature vector for each image \cite{DeepSiameseCNN}. Using a similarity criterion, a new feature vector is created using the two feature vectors and a fully connected layer and a sigmoid activation function creates the predicted similarity score. This approach yielded an average test accuracy of $67.66\%$ over all of the relationships that were in the dataset. 
Another paper that used a Siamese approach aimed to solve both the standard kinship verification problem and the Tri-Subject problem \cite{DeepFusion}. Both networks for the problems had three stages, a feature extraction stage, which used the ResNet50 or SENet50 models pretrained on VGGFace2 to extract the features from each face into a vector, a feature fusion stage which combines the feature vectors together, and a similarity quantization stage to get the similarity score. 
They then use a jury system to fuse together models which allows them to achieve an accuracy of $75.9\%$. 

However, a prominent approach that is taken includes using metric learning to solve the problem. For example, the paper on Neighborhood Repulsed Metric Learning \cite{KinFaceCitation1} uses a supervised metric learning approach. Using the approach, they aim to find a metric which minimizes the distance between the vectors of images that have a kinship relationship and maximize the distances between the vectors of pairs of images that don't have a kinship relationship. 
Furthermore, the paper that will be implemented in this dissertation, using Weighted Graph Embedding-Based Metric Learning, involves a metric learning approach to the problem \cite{WGEML}. 

\section{Project Overview}

The project deals with a 2019 method using Weighted Graph Embedding-Based Metric Learning \cite{WGEML}, or WGEML for short, and aims to implement the paper and verify the accuracies that were obtained. 
The following is shown in the dissertation:

\begin{enumerate}
    \item The algorithm, WGEML, is implemented and accuracies that are within an acceptable range are obtained, as shown in section \ref{MainResults}.
    \item Ablation studies on the face descriptors are performed and the results are in section \ref{FDAbStudies}.
    % \item Ablation studies are done on the image by blocking certain sections of the image, the results of which are in section \ref{ImageAbStudies}
    \item We use the models that were created to discover biases in the datasets that were used and find out the impact that has overall in section \ref{Biases}. 
    \item We further test the face descriptors by replacing VGG with a smaller CNN in which the implementation is discussed in section \ref{CFNExt}. 
\end{enumerate}

Chapter \ref{Background} explains much of the needed technical background for the implementation, from what a Neural Network is and what metric learning is to each of the face descriptors used and what face descriptors are. 
Chapter \ref{Impl} will then discuss the specifics of how WGEML and the extension that used a different network than VGG was implemented. 
Chapter \ref{Results} then discusses all of the results that were obtained from experimentation with WGEML and the implications of the results. 

%%%%%%%%%%%%%%%%%%%%%%%%%%%%%%%%%%%
%%%%%%%%%%% Preparation %%%%%%%%%%%
%%%%%%%%%%%%%%%%%%%%%%%%%%%%%%%%%%%

\chapter{Preparation} \label{Background}

\section{Convolutional Neural Networks}

We must first 

\subsection{Convolutional Layers}

\section{Face Detection}

\section{Face Descriptors}

We wish to be able to create a mapping from a colored image into a vector to make computations easier and to be able to determine similarity between images, which we do using the distance. These are called image descriptors for general images. When we try and map face images to vectors, we call these face descriptors. 
We want these face descriptor mappings to be able to match the same person in different poses and illuminant geometries to face descriptors that are close to each other in distance. Thus, these mappings try and capture color, texture, and shapes,
for example. There are multiple face descriptors that can be made of a face image, each of which extracts different features from the face. We use the Local Binary Patterns, Histogram of Gradients, Scale-Invariant Feature Transform and VGG face descriptors to extract features from each face.

\subsection{Local Binary Patterns}

One such face descriptor is the Local Binary Patterns (LBP) visual descriptor which is adapted for faces. LBP is a texture descriptor \cite{LBPFace} which means that the algorithm attempts to describe the texture of the image, rather than anything to do with the color. 
As such, given an image we want the descriptor of, we must first convert it to grayscale. Then, for each pixel in the image, a neighborhood of pixels is obtained from it. There are multiple ways to define this neighborhood, as shown in figure \ref{LBPNeighborhood}. 

\begin{figure}[h]
    \centering
    \includegraphics[scale=0.2]{Lbp_neighbors.png}
    \caption{Potential neighborhoods of the pixel}
    \label{LBPNeighborhood}
\end{figure}

The simplest neighborhood, which is the neighborhood used in this project, is the direct neighbors of the pixel, which is the first neighborhood in figure \ref{LBPNeighborhood}. 
However, on the edges, some of the neighbors won't exist, due to the fact that it's on the edge of the image. To mitigate this, in the implementation, I pad the grayscale image with 0s on the outside of the image such that each pixel in the image has the same number of neighbors. 
Once we have our neighborhood of the pixel, we compare the grayscale value of the main pixel with each of the grayscale values in the neighborhood. For each neighboring value, if it is greater than the main pixel, we make it a $1$, otherwise we make it a $0$. 
We are then able to create a binary number from the neighboring values. In practice, where the number is started from doesn't matter but in my implementation, the number starts from the left cell and goes counterclockwise. 
Applying this to figure \ref{LBPExample}, we get that our LBP value for this pixel would be $01111000_2$ which, in decimal, is $120$. 

\begin{figure}[h]
    \centering
    \includegraphics[scale=0.75]{LBPExample.png}
    \caption{Example of LBP operator on a pixel}
    \label{LBPExample}
\end{figure}

To summarize the LBP operator mathematically, given a coordinate in the image, $(x, y)$ which has grayscale value $g$, a neighborhood of $P$ points with radius $R$ enumerated by $g_p$ where $p \in \{0, P - 1\}$, we have that, \cite{textureSurvey}:
\[LBP_{P, R}(x, y) = \sum_{p = 0}^{P - 1} s(g_p - g)2^p\]
Where:
\[s(x) = \begin{cases} 1 & \text{if } x > 0 \\ 0 & \text{else} \end{cases}\]

Once we get the values of the pixel, an added extension, which we do for face description, is to check whether it is a \textit{uniform value} which we define as a value such that, in binary, there are only, at most, 2 bitwise transitions when traversed circularly. For example, 11000111 is a uniform value since it only transitions from 1 to 0 and 0 to 1 but 11001000 isn't since it has 4 bitwise transitions. 
With this extension, we have 58 possible uniform values that a pixel's LBP value can be and an extra value for the LBP value not being uniform. In other words, there are 59 LBP values that a pixel in an image can take. 

Once each pixel in the image has an associated LBP value, we can split up the image into blocks. For our implementation, since our input images have size $64 \times 64$, we split it up into non-overlapping blocks of size $8 \times 8$, of which there are $8 \times 8$ of. 
For each block, we obtain a histogram of the LBP values that were in the block, which gives us a $59$-dimensional vector. To obtain the vector for the entire image, each of these vectors are appended together and a 3776-dimensional LBP face descriptor is obtained for our case. 

\subsection{Histogram of Gradients}

Another face descriptor is Histogram of Gradients (HOG). Given a colored image, we can think of the image as a function, $I: \mathbb{N}_m \times \mathbb{N}_n \to \mathbb{R}^3$, which takes a coordinate in the image and returns a vector of values, which correspond to the red, green and blue values of the pixel.

\subsubsection{Calculating the Gradients}

This then means that we are able to calculate the gradient of the image. This 

\subsubsection{Weighted Vote into Histogram}


\subsection{Scale-Invariant Feature Transform}

\subsection{VGG}

Another way of getting face descriptors for an image is to use the VGG network. The original VGG network \cite{VGG} is a CNN which takes in a $224 \times 224$ colored image and outputs a probability vector of size 1000 for the ImageNet classes. In other words, if \texttt{out} is the output of the model, \texttt{out[i]} corresponds with the probability that the \texttt{i}$^{\text{th}}$ ImageNet class is the class of the input image. 
There are 5 configurations of the network, which have varying depth and, as a result, more parameters. The architecture of the second deepest configuration, which is used in this project, is as follows:

\begin{figure}[h]
    \centering
    T O D O 
    \caption{The Architecture of VGG}
    \label{VGGArch}
\end{figure}

By training this model on the ImageNet dataset, the model is able to get a top-5 classification error of 7.5\%, which means that for 7.4\% of the entries in the test set, none of the top 5 classes that the image could be were the actual image. 

However, in order to use the model for face description, we must change the output layer and the training dataset \cite{VGGFace}. Instead of training on general objects in an image, we train the model on faces, specifically celebrity faces. 
The celebrities are curated to a list of 2622 people in which 2000 images are obtained for each celebrity and the images are then curated. The curated set of images constitutes the training set for the VGG model for faces. 

The VGG model for faces is then the model described above with the softmax layer having dimension of 2622. The model is then trained with the dataset we described which allows us to get rid of the softmax layer and use the output of the last fully-connected layer as our 4096-dimensional face descriptor for the image. 

% \subsection{CifarNet}

\section{Metric Learning}
% Have section on what metric learning is and another paragraph or so to tie it in with the project

We wish to measure the similarity between a pair of faces in order to determine whether they are related or not. One way to approach this is to use imilarity learning, which is a type of approach to the problem in which the goal is to learn a similarity function in order to measure how similar two objects are. 
However, we often use distance to help measure the similarity between data points. By learning what this distance function ought to be, we would be able to find a better similarity function so finding this distance function is the goal of metric learning \cite{metricLearningPrimer}. To go more in depth, we must first define a few concepts.

Given a non-empty set $A$, we define a \textit{distance} over $A$ as a function $d: A \times A \to \mathbb{R}$ such that the following holds:
\begin{enumerate}
    \item $\forall x, y \in A$, $d(x, y) \geq 0$ and $d(x, y) = 0$ if and only if $x = y$. 
    \item \textbf{Symmetry}: $\forall x, y \in A$, $d(x, y) = d(y, x)$
    \item \textbf{Triangle Inequality}: The triangle inequality must hold. That is, $\forall x, y, z \in A$: 
    \[d(x, y) + d(y, z) \geq d(x, z)\]
\end{enumerate}

Given the definition of a distance, we define a \textit{Mahalanobis distance} that corresponds to the matrix $M$ to be a distance function, $d_M: \mathbb{R}^d \times \mathbb{R}^d \to \mathbb{R}$, where $d$ is the amount of dimensions the input vector has, which is defined as:
\[d_M(x, y) = \sqrt{(x - y)^TM(x - y)}\]

In other words, in metric learning, the Mahalanobis distance is used to calculate the distance between data points and the goal of metric learning is to find out what the matrix $M$ should be defined as in order to make data points of the same class closer together and data points of different classes further away. 
We can visualize this as such:

\begin{figure}[ht]
    \centering
    \includegraphics[scale=0.75]{MetricLearning.png}
    \caption{How metric learning works at a high-level view}
    \label{MetricLearning}
\end{figure}

% TODO: Change name
\section{Side Algorithms Used (Temp Name)}

Aside from the main algorithms that will be used which were described earlier, a couple of high-profile algorithms were used that deserve to be explained, although they are used only a few times at most in very specific places. 

\subsection{K-Nearest Neighbors}

K-Nearest Neighbors is an algorithm that finds the $K$ nearest neighbors for each datum in a given set of data. Given a list of datapoints in Euclidean space, we wish to find, for each datapoint, the $K$ closest datapoints to it. 

\subsection{Principle Component Analysis}

Principle Component Analysis (PCA) is a statistical tool that we can use to reduce the dimension of a vector. Firstly, 

\section{Requirements Analysis}



\section{Software Engineering Practices}

\subsection{Starting Point}

Before I started my project, I had knowledge in Python and surface-level knowledge of Keras and Tensorflow. I had also worked with face recognition before. Furthermore, I had used Numpy many times before, so I had 
enough knowledge about it to use it comfortably in my project. However, I hadn't ventured into computer vision before the project aside from knowing very generally what a face descriptor is. 

\subsection{Tools Used}

The project was written in Python 3.6. In order to separate the environment that the project needs with my local environment, I used a virtual environment to contain the installations of the required libraries. I also had a requirements.txt file to contain the names and versions of the packages that were used. 
I also used the following libraries:
\begin{enumerate}
    \item \textbf{Numpy}: Using numpy helped increase performance of many parallel computations, such as matrix multiplication, and provided a simple interface to do these with. 
    \item \textbf{Keras and Tensorflow}: I used Keras to create the VGG model and the CifarNet model. We then use it to train the CifarNet model and then make predictions for both models. The weights of VGG were already pre-computed so no training was necessary for VGG.
    \item \textbf{OpenCV}: I used OpenCV for general image processing, such as converting an image into grayscale or loading in images, and for face detection. 
    \item \textbf{Sklearn}: This library provided an implementation for K-Nearest Neighbors and PCA. 
    \item \textbf{Scipy}: This library provided a function that solved the general eigenvalue problem, given two numpy arrays which was used in the main WGEML algorithm. 
\end{enumerate}

Finally, Git and Github were used for version control and backups. Branches were created for each feature that was to be added to the project and I merged them into the master branch once enough testing was done, whether it was unit testing or integration testing.

\subsection{Datasets}

I used the KinFaceW-I, KinFaceW-II, and TSKinFace datasets to train the model. 

The KinFaceW datasets \cite{KinFaceCitation2} \cite{KinFaceCitation1} are composed of face images from the internet which include public figures as well as their parents or children. Both datasets contain no restriction on pose, lighting background, expression, age, ethnicity, or partial occlusion. 
The main difference between the datasets is that the pairs of face images that have a kin relationship in KinFaceW-I are from different images whereas, in KinFaceW-II, they are from the same image. 
In terms of the specifics of the datasets, they both support four kin relationships, Father-Son (FS), Father-Daughter (FD), Mother-Son (MS), and Mother-Daughter (MD). Each face image are images of size $64 \times 64$. The datasets also contain pre-computed 5 folds for cross-validation for each relationship which contained the names of the pairs of images that had the relationship and those that didn't, henceforth positive and negative pairs respectively. 
Finally, the dataset has 2 main settings which are used: the restricted setting in which only positive pairs of images are used and the unrestricted setting in which negative pairs of images are used. 

The TSKinFace dataset \cite{TSKinFace} contains images for the relationships, Father-Mother-Son (FMS), Father-Mother-Daughter (FMD), and Father-Mother-Son-Daughter (FMSD). 
The dataset contained folders for each relationship and a positive set comprised of the images in which the names of the images had the form ``[relationship]-N-[member].jpg'' in which ``relationship" was the relationship, $N$ was a consistent number and ``member" refers to which member of the relationship they were. For example, ``FMS-10-F.jpg'', ``FMS-10-M.jpg'', and ``FMS-10-S.jpg'' would form a positive triplet. 
The dataset only contained the images in this form so negative pairs of images and the folds for cross-validation had to be computed in the project. 

I examine the effects of the WGEML algorithm on the FS, FD, MS, MD, FMS and FMD relationships, as defined above. 
\subsection{Testing}

Throughout the project, I created unit tests for any modules and for each function in the module. I would only end up merging a feature branch into the master branch if all of the unit tests passed for that feature. 

Along with this, I did integration tests in the form of feeding the scripts small, controlled inputs to see if the scripts would output what was expected. These scripts each used functions from different modules and did a part of the workflow. 

Finally, an end-to-end test was done once each script was tested individually which came in the form of trying the small, controlled input for the first script and then running it through each of the scripts successively. 

\subsection{Licensing}

The SIFT algorithm had been patented in the US. However, the patent expired last year, and the patent was only to protect stop commercial use of the algorithm, whereas this is an academic use of the algorithm.

Furthermore, I am able to use VGG for non-commercial purposes under the Creative Commons Attribution License. 

%%%%%%%%%%%%%%%%%%%%%%%%%%%%%%%%%%%%%%
%%%%%%%%%%% Implementation %%%%%%%%%%%
%%%%%%%%%%%%%%%%%%%%%%%%%%%%%%%%%%%%%%

\chapter{Implementation} \label{Impl}

\section{Repository Overview}

\section{Overview of Workflow}

\section{Face Descriptors}

\section{WGEML}

\section{Prediction}

\section{Data Preparation}

\subsection{Cross-Validation}

\subsection{Positive and Negative Pairs}

\subsection{PCA}

\subsection{Saving Results to Disk}

\section{CifarNet Extension} \label{CFNExt}

%%%%%%%%%%%%%%%%%%%%%%%%%%%%%%%%%%
%%%%%%%%%%% Evaluation %%%%%%%%%%%
%%%%%%%%%%%%%%%%%%%%%%%%%%%%%%%%%%


\chapter{Evaluation} \label{Results}

\section{Success Criterion}

\section{Original Results} \label{MainResults}

\section{Potential Biases in Datasets} \label{Biases}

\section{Ablation Studies}

\subsection{Blocking Face Descriptors} \label{FDAbStudies}

\subsection{Blocking Parts of Images} \label{ImageAbStudies}

\section{Replacing VGG}

\section{Unit Tests}

%%%%%%%%%%%%%%%%%%%%%%%%%%%%%%%%%%
%%%%%%%%%%% Conclusion %%%%%%%%%%%
%%%%%%%%%%%%%%%%%%%%%%%%%%%%%%%%%%

\chapter{Conclusion}


\section{Achievements}

\section{Lessons Learnt}

\section{Future Work}

%%%%%%%%%%%%%%%%%%%%%%%%%%%%%%%%%%%%%%%%%%%%%%%%%%%%%%%%%%%%%%%%%%%%%
% the bibliography
\addcontentsline{toc}{chapter}{Bibliography}
%\bibliography{refs}
\bibliographystyle{plain}
\bibliography{citations.bib}

%%%%%%%%%%%%%%%%%%%%%%%%%%%%%%%%%%%%%%%%%%%%%%%%%%%%%%%%%%%%%%%%%%%%%
% the appendices
\appendix

\chapter{Project Proposal}

%\input{proposal}

\end{document}